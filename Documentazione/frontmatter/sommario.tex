%\selectlanguage{italian}
\begin{abstract}

Il mondo automobilistico è sempre più all'avanguardia grazie all'introduzione dell'elettronica all'interno delle auto. Infatti, con l'introduzione di \emph{ECU}, è possibile garantire elevati standard di sicurezza, minimizzare i consumi di carburante e garantire un elevato comfort a guidatore e passeggeri. Tuttavia, per realizzare dei sistemi in grado di fornire queste caratteristiche, è necessario che le \emph{ECU} siano in grado di comunicare tra loro in maniera affidabile e veloce. Per questa ragione sono stati realizzati dei protocolli che hanno lo scopo di permettere la comunicazione tra le \emph{ECU} e che garantiscono diversi livelli di affidabilità e prestazioni, in maniera tale da adattarsi ad ogni contesto. Anche se i protocolli più diffusi svolgono i propri compiti garantendo diversi standard di affidabilità e prestazioni, lo stesso non vale per la sicurezza rispetto ad attacchi passivi e attivi, infatti, nessuno di questi garantisce un livello di sicurezza adeguato agli standard attuali. Prendendo ad esempio il protocollo \emph{CAN}, questo è vulnerabile ad attacchi di ascolto passivo, attacchi di iniezione e modifica dei messaggi e attacchi di tipo \emph{Denial of Service}, poichè non viene utilizzato nè un meccanismo di cifratura nè un meccanismo di autenticazione dei nodi connessi. Questo, tuttavia, è dovuto al fatto che altrimenti non sarebbe stato possibile garantire le prestazioni necessarie per gestire sistemi molto complessi come quelli \textbf{ADAS}. Dal momento che non vengono prese precauzioni a livello di protocollo, quello che si può pensare, sia per questioni di semplicità che per retrocompatibilità, è di aggiungere uno strato di cifratura direttamente sull'applicativo senza modificare il protocollo, in modo tale da far viaggiare solo messaggi cifrati invece che messaggi in chiaro. Il modo migliore per introdurre questo strato è tramite un cifrario ibrido, il quale permette di far accordare i nodi su una chiave di sessione comune e di contrastare le intercettazioni. Per realizzare il cifrario ibrido si possono usare degli algoritmi \textbf{Post-Quantum}, grazie ai quali si riesce a garantire protezione anche contro computer quantistici ma, ad ogni modo, è importante valutare attentamente le prestazioni di \emph{CAN} con la modifica, in quanto un ritardo troppo alto è inaccettabile e non permette una corretta gestione dei sistemi dell'automobile.
\end{abstract} 